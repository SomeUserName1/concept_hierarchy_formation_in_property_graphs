\chapter{Summary}

In this thesis, we have layed some important groundwork for sophisticated graph
query optimization.

% What have we done
%  theoretical work
%  evaluation
% 
% What are the current limitations
% What is future work

\paragraph{A Graph Query Algebra}

We have formally introduced a query
algebra for graph databases, extending the algebra of Hölsch et al.%
\cite{holsch_algebra_2016} (cf. Chapter \ref{chap:query-algebra}).
We have proven that this graph algebra is well-defined and complete.
Currently, our graph algebra allows to express subgraph matching with label,
type and uniqueness constraints, which is an essential part of any
graph query language.

\paragraph{A Statistical Model to Estimate Result Cardinality}

Moreover, we have illustrated that a global assumption on the distribution of node
labels in a graph database can lead to severe estimation errors for datasets
where this assumption is violated.
We have introduced two new data structures, the \emph{label partition} and the
\emph{sublabel map}, to address this problem.

We have then implemented our own cardinality estimation model in the Neo4j
graph database.
The model provides cardinality estimations for a subset of operators of our
graph algebra, namely for the \op{GetNodes}, \op{Expand} and
\op{NodeLabelSelection} operators
(cf. Section \ref{sec:cypher-to-operators} for the corresponding subset of
Cypher).
For each of these operators, we have explained in detail, which assumptions
lead to the respective estimation function.
Our model uses similar database statistics as Neo4j, plus a label partition and a
sublabel map. Using the information provided by these two data structures, our
model can avoid estimation errors if labels are not independent.
Furthermore, by keeping track of the \emph{label probabilities} at each node
variable of the query pattern, our model is able to deal with strong
dependencies between node labels and the presence of relationships.

We have compared our estimation model to the estimation model of Neo4j
in a statistical analysis. On a catalog of 44 queries about the LDBC Social
Network Benchmark (SNB)\cite{erling_ldbc_2015}, our estimations are
significantly more correlated with the actual result sizes than the
estimations of Neo4j. Our model also has significantly lower relative
estimation errors.

\chapter{Future Work}

\paragraph{Generalizing the Graph Algebra}

We believe that a general graph query algebra is a powerful
mathematical tool to design new graph query optimizers and to analyze the variety
of existing graph databases using a common language.
This motivation has led to the first version of the graph algebra in this thesis.
In the future, we want to extend our algebra to cover more use cases,
e.g. result projections and shortest-paths queries.

\paragraph{Conducting More Tests}

The tests of our model for graph query cardinality estimation have revealed,
that it correlates better with the actual result sizes and generates lower relative
estimation errors than Neo4j's current model on a catalog
of 44 queries about the SNB.
To prove that our model also has the lower absolute estimation errors, we will have
to increase the size of the query catalog.
We will also conduct similar tests on other datasets (e.g. the Movie Database%
\footnote{\url{https://neo4j.com/developer/movie-database/}}) to prove the general
superiority of our estimation model.

\paragraph{Extending the Query Space of the Estimation Model}

Currently, our estimation model is only defined for
a very limited query space (cf. Section \ref{sec:query-restrictions}).
In the future, we want to extend the query space covered by the model by
defining estimation functions for the remaining operators \op{Traverse},
\op{DistinctSelection} and \op{NodeJoin}. The estimation function for the
\op{Traverse} operator requires statistics about the likeliness of cycles in
the database graph. Defining and implementing these statistics is thus a
necessary next step to complete our estimation model.

\paragraph{An Algorithm to Update the Label Partition and the Sublabel Map
           Online}

Our estimation model is informed by a label partition and a sublabel map to
avoid estimation errors if labels are not independent. While we have illustrated
methods to generate these data structures \emph{offline}, it remains an open
research question how they can be updated \emph{online} in the presence of
database updates.
The design of an online algorithm to update the label partition and the
sublabel map is needed to make our estimation model suitable for
update-intensive graph database applications.

\paragraph{Designing the Remaining Parts of the Cascades Query Optimizer}

As mentioned in the Introduction, the estimation model proposed in this thesis
is planned as a part of a new query optimizer for graph databases that will be
implemented in the Cascades framework\cite{graefe_cascades_1995}.
In the future, we want to design and implement the remaining parts of this
optimizer. Firstly, we have to define \emph{physical operators} that are
able to produce the results of combinations of our logical operators in a
specific graph DBMS.
Secondly, we  have to design a \emph{cost model} to estimate the
memory and CPU costs of these operators.
Thirdly, we have to define a complete set of equivalence rules on the
graph query algebra, which can then be used by the Cascades optimizer to
explore the query space.

\paragraph{Combining Research on Graph and Relational Databases}

As a final remark we want to emphasize that,
cardinality estimation in graph databases
and relational databases poses very similar problems. We belief that, much of
the research that has led to todays relational optimizers is also relevant to
graph databases and should not be overlooked. As detailed above, we consider a
general graph query algebra
(motivated by the relational algebra) and a modular, algebraic graph query
optimizer (motivated by the Cascades optimizer for relational databases)
to be promising research areas for the advancement of graph databases.

Inversely, we are also convinced that insights from graph database research can
help to solve current problems of relational databases.
For instance, Guy Lohman states that estimating the
\emph{selectivity of join predicates} is a major issue in  current relational
databases\footnote{\url{http://wp.sigmod.org/?p=1075}, 24.06.2017}.
A relational join can be translated to a relationship matching in a graph database and the
join predicate can often be translated to a constraint on the relationship
type.
Translating the triple statistics maintained in graph databases (number of
relationships of a particular type between nodes having particular labels) to
triple statistics for relational databases (cardinality of a join between two
tables using a particular predicate) could allow to use our graph query
estimation model in a relational database and potentially improve the
state-of-the-art of relational cardinality estimation.

We believe that there are many "low-hanging fruits" for researchers that simply
require some transfer between the relational and graph database communities.
