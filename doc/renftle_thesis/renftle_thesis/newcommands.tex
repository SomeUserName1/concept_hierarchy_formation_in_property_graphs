% Maths
% -----

\newcommand{\lb}{\left(}
\newcommand{\rb}{\right)}
\newcommand{\true}{\texttt{true}}
\newcommand{\false}{\texttt{false}}
\newcommand{\powSet}[1]{\mathcal{P}\left(#1\right)}
\newcommand{\union}{\cup}
\newcommand{\bigunion}{\mathbin{\scalebox{1.5}{\ensuremath{\cup}}}}
\newcommand{\isect}{\cap}
\newcommand{\bigisect}{\mathbin{\scalebox{1.5}{\ensuremath{\cap}}}}

% Natural numbers.
\newcommand{\natnums}{\mathbb{N}}


% Data model
% ----------

% The set of all possible node labels.
\newcommand{\nlabels}{\mathcal{D}_L}

% The set of all possible relationship types.
\newcommand{\rtypes}{\mathcal{D}_R}

% The set of all query variables.
\newcommand{\qvars}{\mathcal{X}}

% Mapping from nodes to corresponding node labels.
\newcommand{\nlabel}{\texttt{l}}

% Mapping from relationships to corresponding relationship type.
\newcommand{\rtype}{\texttt{t}}

% Pairs of uniqueness constraints for relationship variables.
\newcommand{\rpairs}{\texttt{u}}

% Query algebra
% -------------

% The result corresponding to a subgraph pattern.
\newcommand{\result}{\text{res}}

% The space of all query results.
\newcommand{\resultspace}{\mathcal{A}}

% The set of all possible operators.
\newcommand{\operators}{\mathcal{O}}

% Operators.
\newcommand{\op}[1]{\textsc{#1}}

% The GetNodes operator.
\newcommand{\getnodes}[1]{\bigcirc_{#1}}

% The NodeJoin operator.
\newcommand{\join}{\bowtie}

% The Traverse operator (out).
\newcommand{\traverseout}[3]{\tau\!_{#1\, \xrightarrow{#2} \, #3}}

% The Traverse operator (in).
\newcommand{\traversein}[3]{\tau\!_{#1\, \xleftarrow{#2} \, #3}}

% The Traverse operator (variable direction).
\newcommand{\traversevar}[4]{\tau\!_{#1\, \overset{#2}{#3} \, #4}}

% The Expand operator (out).
\newcommand{\expandout}[3]{\varepsilon\!_{#1\, \xrightarrow{#2} \, #3}}

% The Expand operator (in).
\newcommand{\expandin}[3]{\varepsilon\!_{#1\, \xleftarrow{#2} \, #3}}

% The Expand operator (variable direction).
\newcommand{\expandvar}[4]{\varepsilon\!_{#1\, \overset{#2}{#3} \, #4}}

% The NodeLabelSelection operator.
\newcommand{\selection}[1]{\sigma_{#1}}

% The Distinct operator.
\newcommand{\distinct}[2]{\selection{#1 \not = #2}}


% Assumptions
% -----------

% Assumption that node labels form a subset chain.
\newcommand{\chainass}{\ensuremath{\text{D}_\text{chain}}}

% Assumption that node labels are independent.
\newcommand{\indass}{\ensuremath{\text{D}_\text{ind}}}

% Assumption that node labels are disjunct.
\newcommand{\disjass}{\ensuremath{\text{D}_\text{disj}}}

% Assumption that node degrees are uniformly distributed.
\newcommand{\uniass}{\text{U}}

% Assumption on the cardinality of cycles.
\newcommand{\cycleass}{\text{C}}

% Jaccard distance measure between labels.
\newcommand{\jaccdist}{d_J}

% Partition of overlapping labels.
\newcommand{\lpart}{D}

% Strict label disjunctness partition.
\newcommand{\strictlpart}{\widetilde{\lpart}}

% Relation between labels whose node sets overlap.
\newcommand{\overlap}{\text{overlap}}

% Map storing (known) sublabels of a label.
\newcommand{\submap}{s}

% Strict Map storing (known) sublabels of a label.
\newcommand{\strictsubmap}{\widetilde{\submap}}


% Subgraph probability space
% --------------------------

\newcommand{\subgraphs}{\mathcal{S}}

\newcommand{\prob}[1]{\operatorname{P}[#1]}
\newcommand{\sprob}{\operatorname{P}}


% Logical result properties
% -------------------------

\newcommand{\initial}{\textnormal{initial}}
\newcommand{\vars}{\textnormal{var}}
\newcommand{\nvars}{\textnormal{nvar}}
\newcommand{\rvars}{\textnormal{rvar}}
\newcommand{\uniq}[1]{\operatorname{U}[#1]}


% Estimation framework
% --------------------

\newcommand{\props}[1]{\texttt{p}(#1)}
\newcommand{\estprops}[1]{\hat{\texttt{p}}(#1)}
\newcommand{\estfunc}[1]{E(#1)}
\newcommand{\card}[1]{{\left \vert #1 \right \vert}}
\newcommand{\argmin}{\operatornamewithlimits{argmin}}
\newcommand{\avgdeg}{\overline{\operatornamewithlimits{deg}}}


% Evaluation
% ----------

\newcommand{\snb}{\texttt{SNB}}
\newcommand{\mdb}{\texttt{MDB}}

\newcommand{\corr}{\texttt{corr}}

\newcommand{\absErr}{\epsilon}
\newcommand{\relErr}{\eta}

