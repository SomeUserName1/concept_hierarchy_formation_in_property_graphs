\begin{center}
  \Large
  \textbf{A Statistical Model to Estimate Result Cardinality
          in a Graph Database}
\end{center}

\begin{minipage}{\linewidth}
\begin{abstract}
In this thesis we address two issues in the field of graph databases.

Firstly, the relational algebra is a general and precise mathematical
language to design and analyze relational database systems.
We argue that, the graph database community needs a commonly accepted
graph query algebra for the same purposes.
We define a graph query algebra that can be used to express
\emph{pattern matching} queries with constraints on relationship types,
node labels and relationship uniqueness.
Our algebra covers most of the pattern matching available in the popular
graph query language Cypher.

Secondly, current graph databases often do not keep their promise of
outperforming a mapping to a relational database, because their
query optimizers are less sophisticated.
As a first step towards better graph query optimizers,
we develop a new model to estimate result cardinality in a graph
database based on our new graph query algebra.
Our model uses two helper data structures to inform itself about the
distribution of node labels in the database, avoiding the estimation errors
of models assuming independence between all node labels.
In addition, the model performs well even if there are strong dependencies
between labels and the presence of relationships in the database.
In an experimental analysis we demonstrate the superiority of our model
compared to Neo4j's current estimation model.
\end{abstract}
\end{minipage}
