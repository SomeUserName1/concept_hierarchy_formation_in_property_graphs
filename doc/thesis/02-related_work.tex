\chapter{Background} 
\section{Definitions}
In order to elaborate on the methods and results, some concepts, terms and notations need to be defined in the following.
\subsection{Graph Database}
A database management system is a  software system that assists users to  maintain and utilize large amounts of data ~\cite{Ramakrishnan:2002:DMS:560733}.
A graph database management system, henceforth graph database, is a database management system which uses a graph structure as model to represent data~\cite{neo4j_book}, e.g. the Resource Description Framework or the Property Graph Model. Fig. \ref{fig:neo4jarch} shows the high level architecture of the graph database Neo4J. \\
\marginfigure{img/neo4j_arch.png}{neo4jarch}{High level architecture of the Graph database Neo4J. \cite{neo4j_book}}{-6cm}


\subsection{Property Graph Model}
The Property Graph Model is a 9-Tuple $G = (V, E, \lambda, P, T, L, f_P, f_T, f_L)$ with 
\begin{itemize}
    \item $V$ the set of vertices of the graph.
    \item $E \subseteq (V \times V)$ the set of edges of the graph.
    \item $\lambda: E \rightarrow $ a non-reflexive\note{i.e. the edges are directed. \\
    If the graph was directed $\lambda$ would be a reflexive function. \\
    Normally in a graph the edges $E \subseteq (V \times V)$ but in the property graph model edges have sets of properties, thus making them objects on their own.}
 function assigning a pair of nodes to an edge.
    \item $L$ a set of strings used as labels.
    \item $P$ a set of key-value pairs of type String, Value\note{the actual supported types of values depend on the implementation} called properties.
    \item $T$ a set of strings used as relationship types.
    \item $f_P: V \cup E \rightarrow 2^P$ a function that assigns a set of properties to a node or relationship.
   \item $f_T: R \rightarrow T$ a function that assigns a type to  a relationship.
   \item  $f_L: N \rightarrow 2^L$ a function that assigns a node a set of labels.
\end{itemize} 
\smallskip
In words: \\
The property graph model reflects a directed, node-labeled and relationship-typed multi-graph $G$, where each node and relationship can hold a set of key-value pairs \cite{angles2018property}.


\subsection{Taxonomy}
Taxonomies\note{sometimes also referred to as concept hierarchy} organize observations into hierarchical classification schemes. A Taxonomy groups a set of objects depending on their properties and are able to represent sub- and super-ordinations as well as inheritance. An example are biological taxonomies, grouping animnals and plants into domains, Kingdoms, Phyla, Classes, and so on~\cite{Krcmar2015}~\cite{han2011data}. A Taxonomy can also be seen as a hierarchy of labels, associated with certain concepts. \\
\marginfigure{img/taxonomy_ex.png}{taxonomy}{This graph shows the main taxonomic ranks in biology \cite{TaxonomicRankGraph}}{-5cm}


\subsection{Formal Concept Analysis}
In order to formalize the description of Objects holding key-value pairs one can use fomral concept analysis, developed by Wille~\cite{wille1982restructuring}. Based on lattice theory, formal concept analysis defines a mathematical framework to exactly describe how objects and attributes may be ordered and visualized by Hasse diagrams. The following terminology is a summary of \cite{ganter2012formal}. \\

Let $O, A$ be sets, $I \subseteq (O \times A)$ a relation. We call $O$ the set of objects, $A$ the attributes set and $I$ the identity relation, assigning attributes to an object (or the other way round specifying all objects that have a certain attribute). \\

Then the formal context $\mathcal{C}$ is given by the triple $\mathcal{C} = (O, A, I)$. A Context is easily visualizable by a table as in Fig.\ref{fig:tablehasse}. \marginfigure{img/table_hasse.png}{tablehasse}{Tabular representation of a formal context}{-4cm} \\

Let $O_0 \subseteq O$ be a subset of all objects and $A_0 \subseteq A$ be a subset of all attributes then:
\[ A' = \{ a \in A | \forall o \in O_0: oIa \}  \]
the set of attributes, that the objects in $O_0$ have in common and 
\[ O' = \{ o \in O | \forall a \in A_0: oIa \} \]
the set objects having all attributes in $A_0$. \\

A formal Concept of the context $\mathcal{C}$ is a pair $(O_0, A_0)$ with $O_0 \subseteq O$,  $A_0 \subseteq A$ and $O_0 = O'$ with respect to $A_0$ and $A_0 = A'$ with respect to $O_0$.

%%
%% TODO: Unterbegriff, Oberbegriff, Partitionierung
%% Concept lattice
%%% Multi-valued contexts
%% Hierarchical ordering


Fig. \ref{fig:hasse} shows the Hasse diagram for the example context from the table above. 
\marginfigure{img/hasse.png}{hasse}{A Hasse-diagram for the above table}{-1cm}



\subsection{Cluster Analysis}
Clustering or Cluster Analysis is defined by the automated process of splitting the set of observations into subsets, in order to partition them such that objects in different clusters are different from another and objects within a cluster are as similar. \\
Let $O$ be the set of Objects. \\
Let $O_0, O_1, \dots, O_n$ with $\forall i \in \{0, 1, \dots n\} O_i \subset O \wedge \cup_i O_i = O$. Then $O_0, O_1, \dots, O_n$ is a partition. \\
Each subset $O_i$ of the partitioning is called a cluster and Clusters may overlap.
Depending on the method used and the measure optimized for, the clusters differ between different algorithms.~\cite{han2011data}. \\
Thus clustering may be defined as a search for a certain partitioning of the input, satisfying some condition or optimizing for a certain metric function e.g. intra-cluster homogeneity and inter-cluster diversity~\cite{Fisher1987}. \\
In order to compute any metric or find similarities and differences or patterns in the objects, the objects must have some attributes\note{If no object has an attribute the only logic partitionings are the trivial ones: \begin{itemize}
    \item $O_0 \equiv O \wedge \forall i \in \{1, \dots n\}: O_i \equiv \emptyset$
    \item $\forall i \in \{0, 1, \dots n\}: |O_i| = 1 \wedge \sum_i |O_i| = |O|$
\end{itemize}}.
Thus all algorithms can be defined by the following interface:
\begin{algorithm}
    \KwIn{Set of Objects $O$ with Attributes $A$}
    \Parameters{Parameters P to the implementation if necessary}
    \KwOut{A partitioning of $O: O_0, O_1, \dots, O_n$} 
\caption{Clustering Algorithm}\label{clustering}
\end{algorithm}

\subsection{Hierarchical Clustering}

\subsection{Conceptual Clustering}

 



\section{Related Work}
\subsection{Clustering Surveys}

\subsection{Graph Database-oriented Conceptual Clustering Methods}
\subsubsection{Incremental Self-Organizing Memory}
\subsubsection{Subdue}

\subsection{Feature Extraction}
\subsubsection{Characteristic Sets}
\subsubsection{Recursive Feature Extraction}
\subsubsection{Role Extraction}
\subsubsection{Aggregation-based Feature Invention}

